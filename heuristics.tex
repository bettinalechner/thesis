\begin{itemize}
	\item \textbf{Visibility of system status}
		\begin{itemize}
			\item The system should always keep users informed about what is going on, through appropriate feedback within reasonable time.
			\item Are users kept informed about system progress with appropriate feedback within reasonable time?
			\item Provide timely feedback about all processes, system status.
		\end{itemize}
		
	\item \textbf{Match between system and the real world}
		\begin{itemize}
			\item The system should speak the users' language with words, phrases, and concepts familiar to the user, rather than system-oriented terms. Follow real-world conventions, making information appear in a natural and logical order.
			\item Does the system use concepts and language familiar to the user rather than the system-oriented terms? Does the system use real-world conventions and display information in a natural and logical order?
			\item Speak the user's language (avoid jargon). Make use of existing knowledge (familiar mental models).
		\end{itemize}
		
	\item \textbf{User control and freedom}
		\begin{itemize}
			\item Users often choose system functions by mistake and will need a clearly marked ``emergency exit'' to leave the unwanted state without having to go through an extended dialogue. Support and undo and redo.
			\item Can users do what they want when they want?
			\item Make sure the user can get out of an undesirable state easily. Design assuming that people will make errors and need to recover previous states.
		\end{itemize}
		
	\item \textbf{Consistency and standards}
		\begin{itemize}
			\item Users should not have to wonder whether different words, situations, or actions mean the same thing. Follow platform conventions.
			\item Do design elements such as objects and actions have the same meaning or effect in different situations?
			\item Make sure the same term/action has one meaning. When there is no better way, conform to a standard.
		\end{itemize}
		
	\item \textbf{Error prevention}
		\begin{itemize}
			\item Even better than good error messages is a careful design which prevents a problem from occurring in the first place.
			\item Can users make errors which good designs would prevent?
			\item Make it difficult to make errors.
		\end{itemize}
		
	\item \textbf{Recognition rather than recall}
		\begin{itemize}
			\item Make object, actions, and options visible. The user should not have to remember information from one part of the dialogue to another. Instructions for use of the system should be visible or easily retrievable whenever appropriate.
			\item Are design elements such as objects, actions and options visible? Is the user forced to remember information from one part of a system to another?
			\item Remove the need to remember across dialogues. Provide multiple views for easy comparisons.
		\end{itemize}
		
	\item \textbf{Flexibility and efficiency of use}
		\begin{itemize}
			\item Accelerators---unseen by the novice user---may often spe\-ed up the interaction for the expert user to such an extent that the system can cater to both inexperienced and experienced users. Allow users to tailor frequent actions.
			\item Are task methods efficient and can users customise frequent actions or use short cuts?
			\item Provide shortcuts (quick keys, customization).
		\end{itemize}
		
	\item \textbf{Aesthetic and minimalist design}
		\begin{itemize}
			\item Dialogues should not contain information which is irrelevant or rarely needed. Every extra unit of information in a dialogue competes with the relevant units of information and diminishes their relative visibility.
			\item Do dialogues contain irrelevant or rarely needed information?
			\item Avoid extraneous information, steps, and actions. Information should be in a logical, natural order.
		\end{itemize}
		
	\item \textbf{Help users recognize, diagnose, and recover from errors}
		\begin{itemize}
			\item Error messages should be expressed in plain language (no codes), precisely indicate the problem, and constructively suggest a solution.
			\item Are error messages expressed in plain language (no codes), do they accurately describe the problem and suggest a solution?
			\item Diagnose the source and cause of a problem and suggest a solution.
		\end{itemize}
		
	\item \textbf{Help and documentation}
		\begin{itemize}
			\item Even though it is better if the system can be used without documentation, it may be necessary to provide help and documentation. Any such information should be easy to search, focused on the user's task, list concrete steps to be carried out, and not be too large.
			\item Is appropriate help information supplied, and is this information easy to search and focused on the user's tasks?
			\item Design for use without documentation. Provide easy-to-use task-oriented help.
		\end{itemize}
		
	\item \textbf{Navigation}
		\begin{itemize}
			\item Information can be easily accessed
			\item Functionality can be found quickly and easily
			\item The system can guide the user through the correct sequence of transaction to complete a business process
			\item The UI supports efficient and accurate navigation of the system
			\item Functionality to search for information that is available
			\item There is a correlation between the searched item and the required information
			\item The system is capable of supporting the different interaction styles of the various users
			\item The system supports alternative navigation metaphors
			\item The system supports guidance-type information
			\item There is clarity in terms of the next sequence of transactions of steps
		\end{itemize}
	
	\item \textbf{Presentation}
		\begin{itemize}
			\item The visual layout is well designed
			\item The information provided by the system is timely, accurate, complete and understandable
			\item The output is easy to understand and interpret, whether the output is structured
			\item The information presented supports informed decision making
			\item The output provided provides clear visibility into the various other departments
			\item The UI is intuitive
			\item Maintain UCD [user-centered design] attributes for interface graphical aspects
			\item Introduce mechanism to highlight errors and cues to avoid errors
			\item Provide the possibility to personalize interface graphics
			\item Clearly and constantly indicate system state
			\item Clearly visualize progress tracking
			\item Clearly visualize options and commands available
			\item Clearly visualize course structure
			\item Provide adaptation of the graphical aspect to the context of use
		\end{itemize}
		
	\item \textbf{Task Support}
		\begin{itemize}
			\item The terminology used by the system is consistent with the terminology of the user
			\item The information provided by the system is in real-time
			\item The responses from the system are quick and efficient
			\item The system supports efficient completion of tasks
			\item The system improves user productivity
			\item The system automates routine and redundant tasks
			\item The system is easy to use
			\item The system supports improved information flow between the various organizational departments
		\end{itemize}
		
	\item \textbf{Learnability}
		\begin{itemize}
			\item A user can learn how to use the system without a long introduction
			\item The various functions of the system can be identified by exploration
			\item There is sufficient on-line help to support the learning process
			\item It is easy to become skillful at using the system within a short amount of time
			\item The system is intimidating and complex to learn and use
		\end{itemize}
		
	\item \textbf{Customization}
		\begin{itemize}
			\item The ease in which the system can be configured to a particular industry type
			\item The capability of the system to support user-level customization
			\item The capability of the system to support customization for the user at a transaction level
			\item The alignment of the system to update existing business processes, and (or) to include new ones
			\item The ability of the system to be re-configured over a period of time
			\item The ability of the UI to be configured without affecting the underlying business logic of the system
		\end{itemize}
		
	\item \textbf{Hypermediality}
		\begin{itemize}
			\item Provide support for the preparation of the multimedia material
			\item Highlight cross-references by state and course maps to facilitate topic links
			\item Supply different media channels for communication
			\item Maximize personalized access to learning contents
			\item Allow repository access to both lecturer and student
			\item Create contextualized bookmarks
			\item Enable off-line use of platform maintaining tools and learning context
		\end{itemize}
		
	\item \textbf{Application Proactivity}
		\begin{itemize}
			\item Insert assessment tests in various forms
			\item Automatically update students' progress tracking
			\item Insert learning domain tools
			\item Provide mechanisms to manage user's profiles
			\item Introduce mechanisms to prevent usage errors
			\item Provide mechanisms for teaching-through-errors
			\item Allow different repository modes for lecturers and students
			\item Insert easy to use platform tools
			\item Maximize adaptation of technology to the context of use
			\item Register the date of last modification of documents to facilitate updating
		\end{itemize}
		
	\item \textbf{User activity}
		\begin{itemize}
			\item Provide easy-to-use authoring tools
			\item Enable to define a clear learning path
			\item Allow to define alternative learning paths
			\item Provide support for assessment test
			\item Manage reports about attendance and usage of a course
			\item Allow use of learning tools even when not scheduled
			\item Provide both synchronous and asynchronous communication tools
			\item Provide communication mechanisms to both students and lecturers
			\item Allow the possibility to personalize the learning path
			\item Insert mechanisms to make annotations
			\item Provide mechanisms to integrate the didactic material
			\item Provide mechanisms for search by indexing, key or natural language
			\item Allow the possibility to create standard-compliant documents and tests (AICC, IMS, SCORM)
			\item Provide authoring tools to facilitate documents updating and assessment tests editing
		\end{itemize}
\end{itemize}