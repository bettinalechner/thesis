\chapter{Discussion \& Conclusion}
\label{chap:discussion}
The purpose of this study was to identify the usability heuristics relevant to web-based CRM systems. To investigate this problem, a two-phased study was conducted involving both usability experts and CRM users. While the previous chapter presented the results of the two phases as well as the comparative analysis, this chapter presents a discussion of the results and the findings gleaned from them, as well as their implications and the study's limitations.

\section{Discussion}
\subsection{Applicability Based on Specificity}
The results of the first phase showed a gradation in applicability of the heuristics based on their source. The most general set of heuristics \citep{Nielsen1994a} received the highest ratings at 4.087, followed closely by the heuristics specifically developed for ERP systems \citep{Singh2009} at 4.086 points, while the heuristics for e-learning systems \citep{Ardito2006} received the lowest ratings at 3.647 points. The reason for this may be that the general heuristics were not developed with a specific class of applications in mind. They were intentionally created to be applicable to a wide array of user interfaces \citep[p.\ 28]{Nielsen1994a}.

On the other hand, previous research has shown that some usability problems may be missed, if only general heuristics are used in an evaluation \citep[e.g.][]{Rusu2011}, so it is necessary to create heuristics for specific applications to improve usability evaluations, even though they may not be widely applicable.

\subsection{Applicability Based on Domain}
The ERP heuristics were likely rated as more applicable than the e-learning heuristics, because ERP systems are more similar to CRM systems. In fact, many ERP systems contain CRM systems. They also support similar work flows and users, as opposed to e-learning systems, which have a different audience and goals. ERP and CRM systems are usually used to support and even automate business processes, while e-learning systems are used to support learning processes and education.

This finding shows that similar classes of applications share more of the same heuristics than applications used for very different purposes. This finding also supports previous research indicating that different heuristics are needed to appropriately cover different classes of applications \citep[e.g.][]{Rusu2011}. In addition, the \textit{Task-Technology Fit} model discussed in section~\ref{sec:ttf} shows that task characteristics have a direct influence on task-technology fit. Two tasks with very different characteristics will likely also have a different technology fit, whereas tasks which share many characteristics and are more similar will have a similar fit to a technology.

\subsection{Lack of Specificity of Usability Experts' New Heuristics}
The five new heuristics developed by the usability experts in phase one are relatively general and do not use any domain-specific terminology. They appear to be applicable to a variety of systems that are used to store and access large volumes of information and support business processes.

The reason for this lack of specificity may be that most of the usability experts had very little experience with CRM systems. On average, the participants had only 1.2 years of experience using a CRM system. \Citet{Nielsen1992} found that in heuristic evaluations, evaluators who have expertise in both usability as well as the subject matter of the system being evaluated find 19 percentage points more usability problems than evaluators who do not have subject matter expertise. It is quite likely that the participants in this thesis study had difficulty developing heuristics that are more specific to CRM systems because they were not ``double specialists''. This limitation is discussed in greater detail in section~\ref{sec:limitations}.

\subsection{Differences in Ratings for \citeauthor{Nielsen1994a}'s Heuristics}
On average, the usability experts rated the heuristics developed by \citet{Nielsen1994a} higher than the CRM users did. The average applicability rating assigned to \citeauthor{Nielsen1994a}'s heuristics (4.10) by usability experts is 0.81 higher than the average rating for all heuristics (3.29). On the contrary, the average rating assigned to \citeauthor{Nielsen1994a}'s heuristics (4.08) by CRM users is 0.10 points lower than the average rating for all heuristics (4.18). This means that the usability experts considered \citeauthor{Nielsen1994a}'s heuristics more applicable and important than the CRM users did.

Based on observations made during the focus group discussion, it is likely that most of the usability experts recognized these heuristics and their importance in the field of usability research, while none of the CRM users did. Since the usability experts recognized \citeauthor{Nielsen1994a}'s importance in the field, which became evident in the focus group discussion, they may have been biased to assigning a higher rating to his heuristics than they actually deserved.

It is also possible that the CRM users assigned lower ratings to \citeauthor{Nielsen1994a}'s heuristics, because they are general in nature and the participants therefore thought they were less useful than the more specific heuristics.

\subsection{Low Rating for \textit{User Control and Freedom} Heuristic by CRM Users}
One of \citeauthor{Nielsen1994a}'s heuristics (``Users often choose system functions by mistake and will need a clearly marked `emergency exit' to leave the unwanted state without having to go through an extended dialogue. Support undo and redo.'') received a particularly high rating by the usability experts, while the CRM users assigned the lowest rating by far to it.

One possible explanation is the previously mentioned bias of the usability experts to rate this heuristic highly, although it may not be very applicable in reality. This explanation seems possible, yet is unlikely, because there is such a big discrepancy between the two ratings.

Another explanation might be that the users took offense to the wording of the heuristic, thinking that they do not, in fact, make a lot of mistakes when using a system. It is also possible that the users had trouble understanding this particular heuristic, as they assigned higher ratings to other heuristics related to error avoidance and recovery (``Introduce mechanisms to prevent usage errors'' received a rating of 4.6, ``Introduce mechanisms to highlight errors and cues to avoid errors'' received a rating of 3.0). Further research is necessary to investigate this discrepancy.

\subsection{Answers to Research Questions}
The first research question, ``What are the usability heuristics relevant to web-based CRM systems?'' was answered by the creation of the unified list in table~\ref{tab:final_list}. Existing heuristics were evaluated by two sets of participants and those with high ratings of applicability to web-based CRM systems were included in the final list. It contains heuristics from four sources: the heuristics created by \citet{Nielsen1994a}, \citet{Singh2009}, and \citet{Ardito2006}, as well as heuristics created by the participants of the study.

The answer to the second research question, ''Which heuristics are irrelevant?'', shows that heuristics are irrelevant to a particular class of applications, if they contain domain-specific wording or references to user interface elements, which are not used in the class of applications at hand. More broadly-formulated, general heuristics are applicable to more classes of applications, but less helpful in identifying domain-specific usability problems.

The final research question, ``Are there new heuristics that need to be added?'', can be answered with a ``yes''. Seven new heuristics were created during the study and added to the body of heuristics for web-based CRM systems. It is expected that these heuristics are most relevant to web-based CRM systems and less relevant to other types of computer systems.

\section{Contributions}
This study was the first to investigate usability heuristics for web-based CRM systems as well as the effects of the involvement of users in the process of developing domain-specific usability heuristics. It was shown that there is a difference in the applicability of a set of heuristics based on the domain it is applied to. General heuristics remain largely applicable to specific domains, while the applicability of specific heuristics depends on the similarity of the domain they were developed for and the one they are used to evaluate.

The findings of this study demonstrate that users can be involved in the process of developing domain-specific heuristics and can add subject matter expertise, especially if the usability experts have no or only limited experience with the specific class of applications. It was shown that people unfamiliar with usability evaluations and heuristics are still able to understand them for the most part and can make valuable contributions and judgments on their applicability to a domain they are familiar with.

\section{Implications}
\subsection{For Research}
The results of this study show that there is a need for developing domain-specific usability heuristics for the various classes of software in use. It will also be necessary to periodically evaluate existing heuristics to investigate whether they need to be adapted to account for emerging types of systems.

Since this study found that even people untrained in usability evaluations can understand heuristics and make contributions to the development of them, the question of whether usability experts are still necessary arises. Traditionally, usability experts both create usability heuristics and then use them to evaluate software and identify usability problems. Now that non-experts were successfully included in the development of new heuristics, it remains to be answered whether this holds true in general and usability experts do not have to play the main role in developing new heuristics, but can inform the process and support domain experts in the development. It is also possible that the non-experts could perform a heuristic evaluation by themselves and apply a set of defined heuristics correctly to identify usability problems.

There are also implications with regard to the two theories discussed in section~\ref{sec:theories}. Some of the constructs contained in these models have relationships with the concepts and findings developed in this thesis.

\subsubsection{IS Success Model}
The study performed in this thesis mostly relates to the independent variables in the IS Success Model, as these capture the quality of the system, information, and service, rather than the attitudes and behaviors which are captured by the dependent variables. Usability heuristics are used to improve the quality of an information system, which will then influence the attitudes and behaviors.

\paragraph{System quality} measures the desired characteristics of the information system itself \citep{DeLone2004}. Among these qualities are usability, availability, and reliability. \textit{System flexibility} is also one of the characteristics included in \textit{system quality} and was originally developed by \citet{Hamilton1981a}. In the final list of heuristics developed in the present study, there are thirteen items related to flexibility and customizability, which is a component of flexibility (see table~\ref{tab:flexibility_heuristics}).

\begin{table}[htb]
	\vspace{0.5cm}
	\centering
	\caption{Heuristics related to system flexibility with overall applicability ratings}
	\label{tab:flexibility_heuristics}
	\begin{tabularx}{\textwidth}{Xcc} \toprule
		\textbf{Heuristic} & \textbf{Source} & \textbf{Applicability} \\ \midrule
		Maximize adaptation of technology to the context of use & A & 3.73 \\
		Provide mechanisms for search by indexing, key or natural language & A & 3.93 \\
		Supply different media channels for communication & A & 3.82 \\
		The system is customizable at the user level & A & 3.00 \\
		The system's communication mechanisms match the needs of the users & A & 3.18 \\
		Accelerators -- unseen by the novice user -- may often speed up the interaction for the expert user to such an extent that the system can cater to both inexperienced and experienced users. Allow users to tailor frequent actions. & N & 3.96 \\
		The system allows for tailoring of the interface to an individual's workflow & New & 3.17 \\
		The system is accessible and usable from mobile devices & New & --- \\
		The ability of the system to be re-configured over a period of time & S & 4.13 \\
		The ability of the UI to be configured without affecting the underlying business logic of the system & S & 3.64 \\
		The capability of the system to support user-level customization & S & 3.55 \\
		The ease in which the system can be configured to a particular industry type & S & 3.89 \\
		The system is capable of supporting the different interaction styles of the various users & S & 3.64 \\
		\bottomrule
	\end{tabularx}
\end{table}

These heuristics generally recommend that the system be flexible to allow for varying interaction modes across users as well as for customization of the system to different users, business processes, and industries. Since there is a total of 70 heuristics in the list, the percentage of heuristics relating to system flexibility is 18.6\%. This finding reinforces the relationship between usability and system quality.

\paragraph{Information quality} measures the quality of the information system's output \citep{DeLone1992}. This construct contains measures such as relevance, understandability, accuracy, and timeliness of the information provided by the system. There are nine items in the final list of heuristics developed in the present study relating to information quality (see table~\ref{tab:iq_heuristics}). The heuristics are concerned with clarity, relevance, usableness, and timeliness of the system's output. This shows that information quality is important for the usability of CRM systems.

\begin{table}[htb]
	\vspace{0.5cm}
	\centering
	\caption{Heuristics related to information quality with overall applicability ratings}
	\label{tab:iq_heuristics}
	\begin{tabularx}{\textwidth}{Xcc} \toprule
		\textbf{Heuristic} & \textbf{Source} & \textbf{Applicability} \\ \midrule
		Clearly visualize employee performance & A & 3.55 \\
		The system displays appropriate information depending on the task at hand & New & 4.17 \\
		The system has a dashboard which provides a quick glance of the current status & New & 4.83 \\
		The system allows for synchronization of its information with outside communication tools & New & --- \\
		Information can be easily accessed & S & 4.55 \\
		The information presented supports informed decision making & S & 4.53 \\
		The information provided by the system is in real-time & S & 4.18 \\
		The information provided by the system is timely, accurate, complete and understandable & S & 4.73 \\
		The output provided provides clear visibility into the various other departments & S & 3.82 \\
		The output style fits the type of data being displayed & S & 4.15 \\
		\bottomrule
	\end{tabularx}
\end{table}

\paragraph{Service quality} is a construct focused on the support the information system and IS organization provides for end users and includes \textit{tangibility}, \textit{reliability}, \textit{responsiveness}, \textit{assurance}, and \textit{empathy} \citep{DeLone2003}. With regard to this thesis, two instruments in this construct are of particular interest, \textit{responsiveness} (promptness of service to users) and \textit{assurance} (knowledge of IS employees to do their job well). During the phone interviews performed with CRM users, one of the participants pointed out that help and documentation are not important to him, because he prefers to consult the IT department when he has a question about or problem with the CRM application instead of reading the help documentation. It seems reasonable to assume that this attitude is fairly common among end users and therefore, a responsive and knowledgeable IT help desk are important in ensuring the success of an information system.

This discovery shows that there is a relationship between service quality and usability, as increased service quality through a responsive and competent help desk can increase the ability of the end users to properly operate the system and therefore increase \textit{user satisfaction} with the system. There is also an interesting implication when looking at the relationship in the opposite direction. A good help desk could effectively erase the need for good documentation of the system for some users, although others may still prefer to use the documentation instead of consulting the help desk.

\subsubsection{Task-Technology Fit}
The results of the study performed for this thesis have implications for the three variables in the task-technology fit model developed by \citet{Goodhue1995}.

\paragraph{Task characteristics} are the characteristics of the ``actions carried out by individuals in turning inputs into outputs'' \citep{Goodhue1995}. Tasks can be classified by their variety and difficulty, interdependence, and routineness \citep{Goodhue1995}. The different tasks performed within CRM systems can be classified into different groups. Answering a customer's complaint call, for example, entails more variety and less interdependence, than creating a report about the sales performance of products across market segments. There are also differences in classification of tasks from different classes of systems. The characteristics of the tasks performed with an accounting system likely differ from those performed with a CRM system.

\paragraph{Technology characteristics} are the characteristics of the ``tools used by individuals in carrying out their tasks''. While an examination of the tools/systems themselves was not part of this study, previous research has examined CRM systems. \Citet{Band2010,Band2010a}, for example, compared the functionality and usability of a variety of CRM systems and found a number of differences.

\paragraph{Task-technology fit} is ``the extent that technology functionality matches task requirements and individual abilities''. The findings of this thesis study show that the more similar the task characteristics are, the more applicable the heuristics are. This finding is in line with the idea that task-technology fit will be similar for tasks which share many characteristics when using the same technology.

\subsection{For Practice}
For practitioners, the findings of this study show that it is indeed important to use both general heuristics and heuristics specific to the class of system being evaluated, as other heuristics are not as applicable. If there is not a set of heuristics specific to the type of system being evaluated, it should be developed. Alternatively, it may be acceptable to use a set of heuristics for a similar type of applications, as this study shows that heuristics for similar types of applications are more applicable than heuristics for very different types of applications. In addition, the set of heuristics developed in this study can be used for heuristic evaluations of web-based CRM systems.

Second, this study has shown that users of the software heuristics are being developed for can be involved in the process successfully. They can add a domain-specific perspective to the development of heuristics, especially when the usability experts do not have that experience. It remains to be answered whether usability experts are even still needed for heuristic evaluations, as it has been shown that users can understand them, although it has not been proven that they know how to apply the heuristics to identify usability problems. This implication could change the role of the usability expert to be someone who informs and guides the development of heuristics and their application.

\section{Limitations}
\label{sec:limitations}
The biggest limitation of this study is its small sample size. Typically, it is recommended to repeat focus groups on a particular topic at least three or four times with different participants to ensure as many points as possible are covered \citep{Edmunds1999,Morgan1998}. Due to time constraints, this was not possible. Instead, the second phase of this study was used to achieve a degree of validity by engaging CRM users.

This means the results only have limited generalizability and thus external validity \citep[p.\ 158]{Creswell1994}. In addition, reproducibility of the results may be limited due to the qualitative and exploratory nature of the study as well as the small sample size, which can lead to decreased reliability \citep[p.\ 159]{Creswell1994}.

With regard to the participants themselves, there were two main limitations. First, the participants in phase one had expertise in usability engineering, but not in using or designing CRM systems (i.e.\ they were not ``double experts''). This means that they had a sub-optimal understanding of the characteristics of CRM systems and the tasks performed on them, which manifested itself during the creation of new heuristics, because their lack of domain-specific knowledge led to the development of less specific heuristics.

Second, the participants of the second study were difficult to recruit. Initially, the researcher planned to perform two focus group sessions (with the same protocol), but due to a lack of participants available at any single point in time, it became necessary to switch to the method described in this study.

This change brought with it another limitation. The participants were not briefed in person about the purpose of the study and their role in it. They merely received a written set of instructions and explanations. It is expected that some participants may not have read these instructions and explanations in full detail and thus did not understand their role in the study as fully as others.

Another limitation during this phase was that in many cases, more than a week passed between the point in time when participants filled out the questionnaire and when they could be reached for the follow-up interview. This means these participants did not have the questionnaire fresh on their minds. To mitigate this limitation, a few engagement questions were asked to refocus the participants on the topic and make them more comfortable expressing their opinion (see section~\ref{sec:session_structure}). Yet, it would have been beneficial to interview all participants within 24 or 48 hours after they filled out the questionnaire.

\section{Future Directions}
\label{sec:future_research}
Due to the limitations faced by this study outlined above, a future research direction is the validation of the results with a greater sample of usability experts and CRM users. This could be accomplished through a method similar to that used in this thesis, e.g.\ through repeated collaborative sessions.

Alternatively, a quantitative methodology for the validation of the results could be developed to be able to capture the judgment of a greater number of participants and analyze the results using more advanced statistical methods. This new methodology could be similar to the questionnaire used in the second phase of this study.

Furthermore, the usefulness of the developed set of heuristics has not been researched. Future research could investigate the usefulness of the heuristics for user interface designers etc.\ seeking to perform a heuristic evaluation of a web-based CRM system.

It would also be interesting to investigate the discrepancy in ratings for the \textit{user control and freedom} heuristic to find out why this heuristic was rated as highly applicable by usability experts and very inapplicable by CRM users.

Another interesting research area would be the success of involving of non-usability experts in the use of heuristics. While this study has shown that non-usability experts can understand heuristics and make valuable contributions in judging their applicability to a domain they are familiar with, it is unclear whether they would be able to apply them in an evaluation to identify usability problems. It would also be interesting to investigate whether heuristics can be successfully evaluated by non-usability experts alone, effectively removing the need for usability experts entirely.

Finally, a generalizable framework for developing usability heuristics for a specific class of applications should be developed, so that new heuristics can be developed more easily and reliably. This framework could be based on the method used in this study.

\section{Conclusion}
\label{sec:conclusions}
This thesis investigated usability heuristics and their applicability to web-based CRM systems. A literature review and research method were presented to address the questions:

\RQ{}

A two-phased, mixed approach combining different qualitative and quantitative methods was employed to answer the research questions.

With regard to the first two research questions, general heuristics were found to be more applicable to this specific class of applications than heuristics developed for other classes of systems which are used for purposes that greatly differ from CRM systems. Heuristics for e-learning systems, for example, were shown to be much less applicable than the general heuristics developed by \citet{Nielsen1994a}.

As for the third research questions, there were seven new heuristics, which were created specifically for web-based CRM systems. This shows that there is a need for domain-specific heuristics for web-based CRM systems.