\chapter{Methodology}
\label{chap:method}
This chapter contains a description of the research methodology used in this thesis. The methodology is built on existing, proven research methods, which were modified and expanded upon to answer the research question central to this thesis.

This study is guided by the following research questions:

\RQ{}

In order to identify new and obsolete usability heuristics specific to the domain of web-based customer relationship management systems, a qualitative design with three steps is employed. A collaborative group session with usability experts is held, followed by a validation through CRM users, after which the results of the two phases are analyzed and compared to existing heuristics.

First, background information about existing methods used in this study is provided. Second, the CRM systems and usability heuristics selected as input materials for the study are discussed. Third, the method for identifying eligible participants is detailed. Fourth, the research design including the structure of the group session is explained. Last, the types of data collected in this study and their sources are detailed.

\section{Background of Methods}
The protocol used for the group session is based on two existing fields of research: focus group studies and collaboration science. These areas and their applicability to this thesis are explained below.

\subsection{Collaboration Science}
The goal of the focus group session with the usability experts was to identify as many new and obsolete usability heuristics for web-based CRM software as possible. To achieve this goal, the session was structured using an approach based on collaboration science and the use of thinkLets \citep{Briggs2001}. This method has proven to build consensus more quickly, reduce session duration, and improve the quality of the end product \citep{Vreede2005}.

While the study is not a collaborative effort (in the strict sense of stakeholders working together to reach a mutually beneficial goal they can commit to), the techniques developed for collaboration science are still helpful in improving the outcome of this undertaking, because the participants are asked to work together and eventually come to a consensus about the list of heuristics for web-based CRM systems.

ThinkLets are ``the smallest unit of intellectual capital required to create one repeatable, predictable pattern of collaboration among people working toward a goal'' \citep{Briggs2003}. ThinkLets allow for the successful collaboration of groups without the need for a highly-trained facilitator by defining reusable process for collaboration in detail \citep{Vreede2005}.

In this study, a group support system (GroupSystems ThinkTank\texttrademark) is used to facilitate the session and instantiate the thinkLet patterns into a collaboration protocol. The first pattern of collaboration used in the group session is a voting activity, in which the participants vote on the heuristics presented to them. The second pattern of collaboration is a brainstorming activity, in which the participants develop ideas for new heuristics. An in-depth description of the structure of the group session is provided in section~\ref{sec:session_structure}.

%\begin{enumerate}
%	\item \textbf{Diverge}: Move from having fewer to having more concepts.
%	\item \textbf{Converge}: Move from having many concepts to a focus on and understanding of a few deemed worthy of further attention.
%	\item \textbf{Organize}: Move from less to more understanding of the relationships among concepts.
%	\item \textbf{Evaluate}: Move from less to more understanding of the benefit of concepts toward attaining a goal relative to one or more criteria.
%	\item \textbf{Build Consensus}: Move from less to more agreement among stakeholders so that they can arrive at mutually acceptable commitments.

\subsection{Focus Group Studies}
Focus group studies have an unclear background and stem from several different research fields such as marketing research, organizational research, and community development \citep[p.\ 18]{Barbour2008}. For this reason, this method has different meanings to different people and is quite flexible. Typically though, focus groups are more or less structured interviews of small groups of people led by a moderator or facilitator, who guides the participants' discussion around a specific topic \citep{Barbour2008,Morgan1998}. Focus group studies have become increasingly popular over the last two decades and are now used in research fields other than those they were used for initially \citep{Morgan1998}.

Rather than providing generalizable answers to questions, focus groups can provide insightful data about how opinions are formed and modified through group interaction as well as an in-depth understanding of the participants' opinions and perspectives \citep{Barbour2008,Edmunds1999}. Focus groups have the advantage of not putting participants ``on the spot'' if they can't answer a question \citep{Barbour2008} and the group interactions allow for the exchange of ideas and opinions between participants.

Due to the ``mixed parentage'' \citep{Barbour2008} of focus groups, many variations exist. A focus group typically consists of seven to ten participants, although ``mini focus groups'' with less participants and bigger groups with more participants have been employed as well \citep{Marshall1999,Edmunds1999}. Other variants exist, in which the focus groups sessions are performed over the phone or Internet \citep{Edmunds1999}. The participants are selected based on shared characteristics considered desirable for the study, such as expertise in a certain area \citep{Marshall1999}. The moderator should be trained and experienced in guiding and stimulating group discussions without influencing the participants.

The focus group method was chosen for the last part of the collaborative group session, wherein the participants were asked to discuss the results and their impressions of the activities completed prior to the focus group session. Table~\ref{tab:focus_group_appropriateness} shows the appropriateness of the focus group method to this study based on the strengths of the method as identified by \citet{Marshall1999}, \citet{Edmunds1999}, and \citet{Morgan1998}.

\begin{table}[hbtp]
	\centering
	\tabulinesep=_2mm^2mm
	\caption[Appropriateness of focus group method to study]{Appropriateness of the focus group method to this study \citep[wording of strengths taken from][]{Marshall1999}}
	\begin{tabu} to \textwidth {X[1,l] X[1,l]} \toprule
		\textbf{Strength of Focus Groups} & \textbf{Application to Study} \\ \midrule
		Fosters face-to-face interaction with participants & Direct interaction between participants can facilitate and accelerate consensus-building \\
		%Data collected in natural setting & \\
		Facilitates immediate follow-up for clarification & Participants can answer questions by the researcher immediately and thus improve data analysis \\
		%Good for documenting major events, crises, social conflicts & \\
		%Useful for describing complex interactions & \\
		%Facilitates discovery of nuances in culture & \\
		Provides for flexibility in formulating hypotheses & Study is exploratory and changes to the research questions are likely \\
		%Provides context information & \\
		Facilitates analysis, validity checks, and triangulation & Results of focus group activity are analyzed in conjunction with quantitative data gathered in the first two activities of the group sessions \\
		Facilitates cooperation & To achieve the ultimate goal of a single, agreed-upon list of heuristics, the participants need to cooperate and come to a consensus \\
		%Obtains large amounts of data quickly & \\
		\bottomrule
	\end{tabu}
	\label{tab:focus_group_appropriateness}
\end{table}

\Citet[p.\ 99ff]{Edmunds1999} explores the advantages and disadvantages of moderating focus groups yourself, which was the case for this study. The biggest advantages are reduced costs by forgoing the need to hire a trained professional from an outside source to moderate and analyze the focus group. In addition, a inside moderator usually also has a better understanding of the area being studied. On the other hand, an inside moderator is often biased and may impart this bias on the participants, knowingly or not. In addition, an untrained moderator can prevent the true results from emerging and thus defeats the purpose of conducting the study \citep{Edmunds1999}.

\section{Participant Selection}
Both of the two phases involved five to six participants. \Citet[p.\ 156]{Nielsen1993} recommends including five to six evaluators in a heuristic evaluation, since a group of this size is expected to find about 75\% of usability problems. Beyond this point, the number of additional problems found per evaluator decreases rapidly \citep{Nielsen1993}. For the purposes of this thesis, it is reasonable to assume that a group of participants of this size can also be expected to identify a large part of heuristics relevant to an application area, although a larger sample size may be required to increase the thoroughness. This limitation will also be discussed in section~\ref{sec:limitations}.

For the first phase, usability experts from both academia and industry were asked to participate. A minimum of three years of experience in designing and/or developing user interfaces and usability was required. Usability experts with experience in using or evaluating web-based CRM software were preferred. The second phase included actual users, who use web-based CRM software in their daily job routines. The participants were required to have at least one year of experience in working with a web-based CRM system, since they should be able to identify usability problems and other characteristics of this type of software that may affect the development of heuristics.

Usability experts were recruited through personal contacts. CRM users were recruited through personal contacts as well as snowball sampling, in which the recruited usability experts were asked to identify and invite CRM users in their organization.

\section{Research Design}
\label{sec:research_design}

\begin{figure}[htbp]
	\centering
	\begin{tikzpicture}
	[ node distance=10mm and 0mm,
		activity/.style={regular polygon,regular polygon sides=6,draw,fill=black!20,text width=1.4cm,text centered},
		document/.style={rectangle,draw,text width=3cm,minimum height=3em,text centered},
		arrowOut/.style={->,shorten >=1pt,>=stealth',semithick},
		arrowIn/.style={<-,shorten <=1pt,>=stealth',semithick}]
			
	\node[document] (materials) {Materials about CRM systems};
	\node[activity] (literature) [above right=of materials] {Lit\-er\-a\-ture Review};
	\node[document] (existing heuristics) [below right=of literature] {Existing Heuristics};
	\node[activity] (group one) [below right=of materials] {Experts Phase};
	\node[document] (expert heuristics) [below=of group one] {Experts' Heuristics};
	\node[activity] (group two) [below=of expert heuristics] {Users Phase};
	\node[document] (user heuristics) [below=of group two] {CRM Users' Heuristics};
	\node[activity] (analysis) [below=of user heuristics] {Final Analysis};
	
	\draw[arrowOut] (literature) -- (materials);
	\draw[arrowOut] (literature) -- (existing heuristics);
	\draw[arrowOut] (materials) -- (group one);
	\draw[arrowOut] (existing heuristics) -- (group one);
	\draw[arrowOut] (existing heuristics) |- (analysis);
	\draw[arrowOut] (group one) -- (expert heuristics);
	\draw[arrowOut] (expert heuristics) -- (group two);
	\draw[arrowOut] (expert heuristics) -| (0,-13.77) -- (analysis);
	\draw[arrowOut] (group two) -- (user heuristics);
	\draw[arrowOut] (user heuristics) -- (analysis);
\end{tikzpicture}
	\caption{Research design}
	\label{img:research_design}
\end{figure}

Figure~\ref{img:research_design} provides an overview of the research design, which consists of four activities based on the development of three sets of heuristics. First, a literature review was conducted. During the literature review, important and relevant usability heuristics were researched and collected. These heuristics served as the basis for the subsequent steps and were eventually used in the final analysis.

Second, a focus group session with usability experts was used to identify heuristics specific to web-based CRM systems. The experts used their judgement to select still-relevant heuristics from the lists identified as part of the literature review, and supplemented them with new heuristics as suitable to cover the characteristics of web-based CRM software.

Third, a validation phase with actual users of a web-based CRM system was completed. This step aimed to increase the internal validity of the results by including the perspectives of users. The participants in this step were given the list of heuristics created in the first phase and were asked to rate the heuristics based on their perceived applicability. After the rating, the participants were asked to identify missing heuristics.

Fourth, the results of the two phases were analyzed to identify the points on which the usability experts and the CRM users agree and on which they disagree. The resulting set of proposed heuristics for web-based CRM systems was then compared to the heuristics used as the initial input for the first focus group. In comparing this new set of heuristics to existing heuristics identified in the literature review, it can then be determined which heuristics have become irrelevant for this class of software, and which new ones need to be added to heuristic evaluations in order to reduce the likelihood of missing problems and to cover as much of the system to be evaluated as possible.

An important consideration is the level of detail at which heuristics are to be evaluated. Some existing sets are high-level and only provide few principles \citep[e.g.][]{Molich1990}, others are detailed and consist of hundreds of items \citep[e.g.][]{Leavitt2006}. For this study, a middle ground was sought by aiming for around fifty to sixty items in the final list. The items were intended to be more specific than stating a general principle like ``consistency'', but less specific than, for example, stating that ``each save button is in the same location''. A heuristic at the right level might be ``buttons are always in the same location''.

\subsection{Pilot}
A pilot study was conducted to help ensure the actual focus group sessions will run smoothly. The goal of the trial run was to uncover potential weaknesses or problems with the procedures so they could be rectified before the actual sessions. The data resulting from the trial run were not used for this study, as the trial's purpose was only to test procedures.

A group of fourteen undergraduate students participated in the trial run. The trial run was performed as a part of the final exam for a user interface design course at the University of Nebraska at Omaha.

For this reason, a number of modifications to the protocol used in the actual sessions had to be made. First, due to the limited amount of time available for the pilot, the students were asked to evaluate only one specific system instead of a class of systems. In addition, an abbreviated list of 26 heuristics was used as the basis for the evaluation. Second, the students were not sufficiently trained to develop new heuristics for the system, so they were only asked to evaluate existing heuristics given to them with regard to applicability. Third, the students were also asked to identify usability problems based on the heuristics supplied to them to satisfy the requirements for the final exam of the class they were taking.

The students were asked to evaluate a student information system familiar to them on the basis of heuristics supplied to them \citep[a modified version of the set developed by][see appendix~\ref{app:trial}]{Oztekin2010}. The students also received a short list of usage scenarios to aid them in purposefully interacting with the system (also in appendix~\ref{app:trial}). Since some of the heuristics were not directly applicable to the system in question, the students were also asked to state whether they believe the heuristics to be appropriate.

After the evaluation phase, the students were asked to discuss the identified usability problems to create a unified list. They were asked to name the usability problems they saw most often in the list they created during the previous activity. The moderator then inserted the problems every student agreed on into a single list in preparation for the final activity. As a last step, the participants rated the usability problems in the list by severity on a three-point scale of ``high'', ``medium'', and ``low''.

During the trial run, no significant problems with the protocol were encountered. The participants were able to understand the instructions and use the group support system easily. Overall, the students also understood the heuristics and were able to evaluate them with regard to their applicability to the student information system used for this pilot.
%results? What type of comments did you get?

The duration of the trial run was limited to one hour, because it was only a part of a exam. This amount of time was far too short to complete the entire protocol. Only 26 heuristics were used and it took most students 45 minutes to comment on each of them. A time frame of one hour for the evaluation component should have provided sufficient time. The subsequent discussion was ended after 15 minutes due to the time constraints mentioned previously. A longer period would have allowed for a more complete discussion. In consequence, a period of three hours was estimated to be necessary to complete the full protocol with the two focus groups.

\section{Research Study}
\subsection{Pre-Tasking}
In preparation for the study, the usability experts were asked to review pre-task material. This is an important step for increasing the level of detail the participants were able to produce, as well as getting them in the right mindset by providing context \citep[p.\ 164]{Cooper2011}.

In order for the participants of the first phase to be able to prepare, the list of existing heuristics from the literature review was provided to them prior to the session for their review. The participants were asked to examine the existing heuristics and start thinking about whether they might be relevant to web-based CRM systems. In addition, screen shots of exemplary systems were sent to the participants to provide visual context and stimuli for developing new heuristics specific to the class of systems at hand.

\subsection{Materials}
\label{sec:materials}
\subsubsection{CRM Systems}
\label{sec:selected_systems}
Based on their leading role in the markets for both large and midsized companies, as well as their wide adoption and availability of documentation, Microsoft Dynamics CRM and salesforce.com were selected as examples of web-based CRM software for this thesis' focus group sessions, which are explained in detail in chapter~\ref{chap:method}. In addition, SugarCRM Sugar Professional was selected, even though it is less present in the enterprise market, because it has the lowest usability rating of the nine leaders and is the only open-source product among them.

\paragraph{Microsoft Dynamics CRM}
Microsoft Dynamics CRM claims to power business productivity through its cloud-based offering. The product can be accessed through a web browser or through a Microsoft Outlook add-in. The software is similar to many of the other Microsoft products in terms of functionality and usability, which makes it familiar to many users \citep{Band2010}. Microsoft Dynamics CRM has the second-strongest presence in the market for enterprise CRM systems, ranking right after Oracle Siebel CRM \citep{Band2010}. Example screen shots of the system are available in appendix section~\ref{appsec:microsoft}.

\paragraph{salesforce.com}
Transforming companies into ``social enterprises'' is the goal of salesforce.com, which they aim to achieve through a strong focus on social media integration and enabling of organization-internal collaboration. The solution is fully web-based and available in a subscription-based software-as-a-service (SaaS) pricing and delivery model, which makes it attractive for organizations not wanting to spend time and money creating an on-premises infrastructure for this purpose. Salesforce.com has the fourth-largest presence in the market for enterprise CRM solutions \citep{Band2010}, and is actually the leader in terms of market presence for midsized companies \citep{Band2010a}. Example screen shots are available in appendix section~\ref{appsec:salesforce}.

\paragraph{SugarCRM Sugar Professional}
As the largest open-source system in the market, SugarCRM is in a special position, since it offers a free edition of its product. Besides the free edition, there are various paid editions with differing levels of service. Paid editions are also available in a hosted delivery model, but all versions can be installed on premises. Due to the open-source status, there is an active developer and support community. Example screen shots are available in appendix section~\ref{appsec:sugarcrm}.

\subsubsection{Usability Heuristics}
The ten heuristics developed by \citet[p.\ 30]{Nielsen1994a} were selected as the basis for the materials for the empirical study, since they are established and well-known general principles.

Two medium-grain sets of heuristics were selected to investigate whether these are considered more or less applicable to the class of systems targeted by this study. One of them was developed for ERP systems, which is a class of systems fairly similar to CRM systems; the other was developed for e-learning systems, which have a quite different set of users and purpose.

The heuristics by \citet{Ardito2006} were chosen to represent the heuristics for systems dissimilar to CRM systems. Finally, the heuristics by \citet{Singh2009} were selected, because ERP systems are similar to CRM systems in terms of users and purpose.

No heuristics at the checklist level were included, because there are too many per set and they are not application-specific. \Citeauthor{Nielsen1994a}'s heuristics were included to represent the general heuristics instead.

\subsection{Session Structure}
\label{sec:session_structure}
The two phases followed two similar structures. Both sessions involved five to six participants and consisted of a quantitative rating activity followed by a qualitative discussion activity.

The differences lie in the input materials, the nature of the qualitative activity, as well as the participants. For the usability expert session, the background materials included a list of existing usability heuristics identified in the literature review. For the CRM user session on the other hand, the materials consisted only of the list of heuristics created by the usability experts in the first session.

Another difference is that the first phase was based on a focus group session, during which all participants collaborated. During the second phase, the participants did not collaborate, but completed the rating and discussion activities individually through an online questionnaire and a phone interview with the researcher respectively.

Figure~\ref{img:session_structure_experts} shows the structure of the usability expert session and figure~\ref{img:session_structure_users} shows the structure of the CRM user session.

\begin{figure}[htbp]
	\centering
	\begin{tikzpicture}
[ node distance=10mm and 10mm,
		activity/.style={regular polygon,regular polygon sides=6,draw,fill=black!20,text width=1.3cm,text centered},
		document/.style={rectangle,draw,text width=3cm,minimum height=3em,text centered},
		arrowOut/.style={->,shorten >=1pt,>=stealth',semithick},
		arrowIn/.style={<-,shorten <=1pt,>=stealth',semithick}]
			
	\node[document] (materials) {Background materials};
	\node[activity] (voting) [below=of materials] {Rating};
	\node[activity] (brainstorming) [right=of voting] {Brain\-storm};
	\node[activity] (discussion) [right=of brainstorming] {Discus\-sion};
	\node[document] (final list) [below=of discussion] {Usability experts' heuristics};
	
	\draw[arrowOut] (materials) -- (voting);
	\draw[arrowOut] (voting) -- (brainstorming);
	\draw[arrowOut] (brainstorming) -- (discussion);
	\draw[arrowOut] (discussion) -- (final list);
\end{tikzpicture}
	\caption{Structure of the collaborative session with usability experts}
	\label{img:session_structure_experts}
\end{figure}

\begin{figure}[htbp]
	\centering
	\begin{tikzpicture}
[ node distance=10mm and 10mm,
		activity/.style={regular polygon,regular polygon sides=6,draw,fill=black!20,text width=1.3cm,text centered},
		document/.style={rectangle,draw,text width=3cm,minimum height=3em,text centered},
		arrowOut/.style={->,shorten >=1pt,>=stealth',semithick},
		arrowIn/.style={<-,shorten <=1pt,>=stealth',semithick}]
	
	\node[document] (first list) {Usability experts' heuristics};	
	\node[activity] (second rating) [below=of first list] {Rating};
	\node[activity] (second discussion) [right=of second rating] {Discus\-sion};
	\node[document] (second heuristics) [right=of second discussion] {CRM users' heuristics};
	
	\draw[arrowOut] (first list) -- (second rating);
	\draw[arrowOut] (second rating) -- (second discussion);
	\draw[arrowOut] (second discussion) -- (second heuristics);
\end{tikzpicture}
	\caption{Structure of the validation phase with CRM users}
	\label{img:session_structure_users}
\end{figure}

\subsubsection{First Phase: Usability Experts}
First, the participants were asked to review the list of heuristics provided to them in the group support system (see appendix~\ref{app:first_group}) and to assign two ratings to each heuristic. The first rating allowed the participants to specify how applicable each of the provided heuristics is to web-based CRM systems in their opinion. This rating was assigned on a five-point numeric scale. The second rating allowed the participants to indicate whether they think any modifications were necessary in order to make a heuristic more applicable (e.g.\ whether wording needed to be changed).

After the first activity, a brainstorming session followed. The participants were asked to re-evaluate the heuristics which received the highest rating for ``need for rewording'' in the previous phase. The participants reworded heuristics were possible and discarded the others. Finally, the participants discussed any ideas for heuristics they thought needed to be added to account for web-based CRM systems as extensively as possible. The group support system also allowed for participants to interact and comment on each others' ideas in addition to the verbal discussion.

\subsubsection{Second Phase: CRM Users}
First, the participants of this phase were asked to review the list of heuristics developed by the usability experts and assign a rating to each. This rating allowed the participants to specify how applicable they think each heuristic is to web-based CRM systems. The rating was assigned on a five-point numeric scale with an additional option of ``don't know'' for cases in which a participant may not understand a heuristic. The questionnaire with the heuristics was made available to the participants in a web-based survey tool.

The researcher then called each participant to administer a follow-up interview. In order to refresh the participants' memory of the usability heuristics, the first questions were ``What do you like about your CRM system?'' and ``What don't you like about your CRM system?''. The participants were then asked to give their overall impression of the applicability of the heuristics they saw in the questionnaire and to identify any areas they felt were not addressed properly by the heuristics. Finally, the participants were asked about the heuristics which were added by the usability experts in order to place a special emphasis on validating their legitimacy.

This phase was conducted with individual interviews as opposed to the focus group approach utilized in the first phase. The main reason for this difference was that the participants in the second phase were less willing to make the bigger time commitment required by the group set-up. \Citet{Robinson1993} found that there is no significant difference in the number and quality of responses obtained through group versus individual interviews. Therefore the change in protocol was not felt to be a problem in answering the research question.

\section{Data Sources and Analysis}
During the study, both quantitative and qualitative data were collected. Quantitative data was collected during the first activity of the two phases, wherein the participants' votes on the applicability of the presented usability heuristics were collected. For the group session, this means data about the participants' opinions of the applicability of existing usability heuristics to the specific context of web-based CRM systems. For the second phase, the votes captured the participants' agreement with the results from the first group session, as the participants' votes indicate their opinions of the applicability of the heuristics developed in the first group session to web-based CRM systems.

Qualitative data was collected during the later activities of each phase. During these activities, the participants provided comments and opinions about the usability heuristics presented to them. During the first phase, the participants engaged in a group discussion about the reviewed heuristics, their need for rewording, as well as any areas missing from the heuristics. During the second phase, the participants engaged in one-on-one phone interviews with the researcher to discuss their overall impression of the heuristics, any missing areas, as well as a targeted discussion of the heuristics newly added by the usability experts.

\subsection{Analysis of Quantitative Data}
Due to the small number of participants in the study, the options for statistical analysis of the data are limited. For the quantitative data collected in the rating activity, descriptive statistics such as means and variance were used to determine overall scores for each heuristic. The heuristics were then sorted based on their applicability score and divided into two groups. The analysis performed on the data is available in sections~\ref{sec:evaluation_existing_first} and~\ref{sec:evaluation_existing_second}.

\subsection{Analysis of Qualitative Data}
Focus group analysis can be challenging and deserves special attention. It is important to note that focus group analysis already starts during the session itself. \Citet{Krueger1998} recommends a systematic approach to focus group analysis consisting of six steps, which were followed while analyzing the data in this study. First, the questions for the participants should be ordered in a way that maximizes insight by allowing for enough time for the participants to order their thoughts and exchange opinions with the other participants. Only then should the actual topical questions be asked. During the focus group, comments should be electronically recorded. In addition, notes should be taken by an assistant moderator, since the main moderator is not likely to have enough time to take thorough notes. At the end of the focus group, it is important to get participant verification, by letting every participant summarize their main points or reading the moderators' notes to the participants and ensuring they agree to them.

Immediately after the focus group, the moderator and other participating personnel should debrief and talk about any important points or quotes they remember. This debriefing should also be recorded electronically for later reference. During the main analysis process, labels should be attached to main ideas or phenomena. These labels or codes are then used as the basis for further analysis such as identifying counts of codes and combinations in which they occur. The final step recommended by \citet{Krueger1998} comes after the analysis: preliminary and final reports should be shared with participants and stakeholders.