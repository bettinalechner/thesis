\chapter{Materials used in Trial Run}
\label{app:trial}

\section{Heuristics}
These heuristics are based on \citet{Oztekin2010}. They were modified to account for the characteristics of the system being evaluated in the trial run by eliminating some heuristics clearly not applicable to the system being evaluated in the trial run.

\begin{singlespace}
		\tabulinesep=_1mm^1mm
		\vspace{0.5cm}
\begin{longtabu} to \textwidth {lX[1, p]}
			\caption{Heuristics used in trial run}\\
			\toprule
			\textbf{Category} & \textbf{Guideline} \\
			\midrule
		\endfirsthead
		
			\bottomrule
		\endlastfoot
		
		Reducing redundancy & Are learning objects easily created and reused? \\
												& Are items visible in multiple places and from multiple paths? \\
												& Does modifying an action or activity require excessive ``redoing'' to make a single change? \\
		Aesthetics 					& Are the screens pleasing to look at? \\
												& Is there proper use of color or graphics that enhance navigation? \\
		Completeness 				& Can you clearly understand all components and structure? \\
												& Is the screen well organized, easy to navigate, and logical? \\
												& Are meaningful labels and descriptive links used to support recognition? \\
		Memorability				& Is there sufficient visibility so the user does not have to look for things and try to remember them? \\
												& Is information presented in organized chunks to support learnability and memorability? \\
												& Is cognitive load reduced by providing familiarity of items and action sequences? \\
												& Is the user offered sufficient FAQ and human support to obtain necessary help? \\
		Consistency, functionality 	& Does the interface provide adequate ``back'' button functionality to return to a previous screen? \\
																& Do the activity, icon, button, label, and links provide clear purpose/intent that matches the tasks? \\
																& Is consistent form and style used for various titles and headers? \\
		Accessibility				& Are alternative pathways to course content and activities available? \\
												& Are accessibility issues addressed throughout the course? \\
												& Are screen features adaptable to individual user preference?\\
		Interactivity, feedback, help	& Is the user provided with sufficient information to know where in the system he/she is? \\
																	& Does the course offer multiple opportunities for interaction and communication among students, to instructor, and to content? \\
		Flexibility 				& Is the speed of loading course page high enough? \\
		Visibility 					& Is the intended functionality clear for each option or selection?\\
												& Are options (buttons/selections) logically grouped and labeled?\\
		Error prevention 		& Can multiple but similar tasks be done easily? \\
												& Can the user easily undo selections, actions, errors in arrangement or management of items? \\
												& Do error or warning messages prevent possible errors from occurring? \\
\end{longtabu}
\end{singlespace}

\section{Instructions for Participants}
\subsection*{Setup}
\begin{itemize}
\item URL: [removed]
\item User Name: [removed]
\item Session ID: [removed]
\item Session Passkey: [removed]
\end{itemize}

\subsection*{Evaluation}
Log into mavlink using your Student ID and password.

Click on ``Search for classes'' on the far left.  Click on ``Use the mavlink search to enroll''.

\begin{figure}[h]
	\centering
	\scalegraphics{./img/other/mavlink}
\end{figure}

As you walk through the following tasks, take note of any usability problems you can identify using the heuristics supplied in ThinkTank.  When you identify a usability problem, leave a comment with a short description of the problem on the corresponding heuristic in ThinkTank.  Comment on every heuristic.  If you can't identify any new problems that relate to a specific heuristic, please state that.

\subsubsection*{Task List}
\begin{enumerate}
	\item Search for all UNO classes offered in the Fall semester of 2012 in the area of Criminology \& Criminal Justice, taught on Mondays and Wednesdays.
	\item You changed your mind and want to take a class that is taught only on Tuesdays instead. You do not want to enroll in a class that is offered on days other than Tuesday (e.g.\ Tuesdays and Thursdays).  All other criteria remain the same.
	\item Search for all Fall 2012 classes taught by Jong-Hoon Youn.
	\item Find out who taught ``INTRO TO WEB DEVELOPMENT'' during Spring 2011.
\end{enumerate}

(The answers to these scenarios are not important.  They are only intended to help you identify usability problems by interacting with MavLink.)