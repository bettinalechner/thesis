\chapter{Lists of Heuristics Created}
This section contains the three lists developed in the various phases of this study. Section~\ref{appsec:litrev_list} consists of the heuristics selected in the literature review, which where presented to the usability experts for review. Section~\ref{appsec:experts_list} contains the heuristics developed by the usability experts during the first phase of the empirical study. This list was used as the basis for the second phase. Section~\ref{appsec:users_list} consists of the heuristics developed by the CRM users during the second phase of the empirical study. Please note that although each heuristic is linked back to its source in the tables following, the participants did not receive this information. They were only presented the heuristics themselves.

\section{List Based on Literature Review}
\label{appsec:litrev_list}

\begin{singlespace}
		\tabulinesep=_2mm^2mm
	\begin{longtabu} to \textwidth {X[1, p] c}
			\caption[List of heuristics developed during literature review]{List of heuristics developed during literature review; N = \citet{Nielsen1994a}, S = \citet{Singh2009}, A = \citet{Ardito2006}}\\
			\toprule
			\textbf{Heuristic} & \textbf{Source} \\
			\midrule
		\endfirsthead
			
			\textbf{Heuristic} & \textbf{Source} \\
			\midrule
		\endhead
		
			\bottomrule
		\endlastfoot
		
		The system should always keep users informed about what is going on, through appropriate feedback within reasonable time. & N \\
		The system should speak the users' language with words, phrases, and concepts familiar to the user, rather than system-oriented terms. Follow real-world conventions, making information appear in a natural and logical order. & N \\
		Users often choose system functions by mistake and will need a clearly marked "emergency exit" to leave the unwanted state without having to go through an extended dialogue. Support and undo and redo. & N \\
		Users should not have to wonder whether different words, situations, or actions mean the same thing. Follow platform conventions. & N \\
		Even better than good error messages is a careful design which prevents a problem from occurring in the first place. & N \\
		Make object, actions, and options visible. The user should not have to remember information from one part of the dialogue to another. Instructions for use of the system should be visible or easily retrievable whenever appropriate. & N \\
		Accelerators -- unseen by the novice user -- may often speed up the interaction for the expert user to such an extent that the system can cater to both inexperienced and experienced users. Allow users to tailor frequent actions. & N \\
		Dialogues should not contain information which is irrelevant or rarely needed. Every extra unit of information in a dialogue competes with the relevant units of information and diminishes their relative visibility. & N \\
		Error messages should be expressed in plain language (no codes), precisely indicate the problem, and constructively suggest a solution. & N \\
		Even though it is better if the system can be used without documentation, it may be necessary to provide help and documentation. Any such information should be easy to search, focused on the user's task, list concrete steps to be carried out, and not be too large. & N \\
		Information can be easily accessed & S \\
		Functionality can be found quickly and easily & S \\
		The system can guide the user through the correct sequence of transaction to complete a business process & S \\
		The UI supports efficient and accurate navigation of the system & S \\
		Functionality to search for information that is available & S \\
		There is a correlation between the searched item and the required information & S \\
		The system is capable of supporting the different interaction styles of the various users & S \\
		The system supports alternative navigation metaphors & S \\
		The system supports guidance-type information & S \\
		There is clarity in terms of the next sequence of transactions of steps & S \\
		The visual layout is well designed & S \\
		The information provided by the system is timely, accurate, complete and understandable & S \\
		The output is easy to understand and interpret, whether the output is structured & S \\
		The information presented supports informed decision making & S \\
		The output provided provides clear visibility into the various other departments & S \\
		The UI is intuitive & S \\
		Maintain UCD [user-centered design] attributes for interface graphical aspects & A \\
		Introduce mechanism to highlight errors and cues to avoid errors & A \\
		Provide the possibility to personalize interface graphics & A \\
		Clearly and constantly indicate system state & A \\
		Clearly visualize progress tracking & A \\
		Clearly visualize options and commands available & A \\
		Clearly visualize course structure & A \\
		Provide adaptation of the graphical aspect to the context of use & A \\
		The terminology used by the system is consistent with the terminology of the user & S \\
		The information provided by the system is in real-time & S \\
		The responses from the system are quick and efficient & S \\
		The system supports efficient completion of tasks & S \\
		The system improves user productivity & S \\
		The system automates routine and redundant tasks & S \\
		The system is easy to use & S \\
		The system supports improved information flow between the various organizational departments & S \\
		A user can learn how to use the system without a long introduction & S \\
		The various functions of the system can be identified by exploration & S \\
		There is sufficient on-line help to support the learning process & S \\
		It is easy to become skillful at using the system within a short amount of time & S \\
		The system is intimidating and complex to learn and use & S \\
		The ease in which the system can be configured to a particular industry type & S \\
		The capability of the system to support user-level customization & S \\
		The capability of the system to support customization for the user at a transaction level & S \\
		The alignment of the system to update existing business processes, and (or) to include new ones & S \\
		The ability of the system to be re-configured over a period of time & S \\
		The ability of the UI to be configured without affecting the underlying business logic of the system & S \\
		Provide support for the preparation of the multimedia material & A \\
		Highlight cross-references by state and course maps to facilitate topic links & A \\
		Supply different media channels for communication & A \\
		Maximize personalized access to learning contents & A \\
		Allow repository access to both lecturer and student & A \\
		Create contextualized bookmarks & A \\
		Enable off-line use of platform maintaining tools and learning context & A \\
		Insert assessment tests in various forms & A \\
		Automatically update students' progress tracking & A \\
		Insert learning domain tools & A \\
		Provide mechanisms to manage user's profiles & A \\
		Introduce mechanisms to prevent usage errors & A \\
		Provide mechanisms for teaching-through-errors & A \\
		Allow different repository modes for lecturers and students & A \\
		Insert easy to use platform tools & A \\
		Maximize adaptation of technology to the context of use & A \\
		Register the date of last modification of documents to facilitate updating & A \\
		Provide easy-to-use authoring tools & A \\
		Enable to define a clear learning path & A \\
		Allow to define alternative learning paths & A \\
		Provide support for assessment test & A \\
		Manage reports about attendance and usage of a course & A \\
		Allow use of learning tools even when not scheduled & A \\
		Provide both synchronous and asynchronous communication tools & A \\
		Provide communication mechanisms to both students and lecturers & A \\
		Allow the possibility to personalize the learning path & A \\
		Insert mechanisms to make annotations & A \\
		Provide mechanisms to integrate the didactic material & A \\
		Provide mechanisms for search by indexing, key or natural language & A \\
		Allow the possibility to create standard-compliant documents and tests (AICC, IMS, SCORM) & A \\
		Provide authoring tools to facilitate documents updating and assessment tests editing & A \\
	\end{longtabu}
\end{singlespace}

\section{List Developed by Usability Experts}
\label{appsec:experts_list}
\begin{singlespace}
		\tabulinesep=_2mm^2mm
	\begin{longtabu} to \textwidth {X[1, p] c}
			\caption[List of heuristics developed during first phase]{List of heuristics developed during first phase; * heuristic is a reworded version of original; N = \citet{Nielsen1994a}, S = \citet{Singh2009}, A = \citet{Ardito2006}}\\
			\toprule
			\textbf{Heuristic} & \textbf{Source} \\
			\midrule
		\endfirsthead
			
			\textbf{Heuristic} & \textbf{Source} \\
			\midrule
		\endhead
		
			\bottomrule
		\endlastfoot
		
		Maximize adaptation of technology to the context of use & A \\
		The system is customizable at the user level & A* \\
		Insert mechanisms to make annotations & A \\
		Clearly visualize user workflow & A* \\
		Clearly visualize options and commands available & A \\
		Clearly and constantly indicate system state & A \\
		Clearly visualize progress tracking & A \\
		Introduce mechanism to highlight errors and cues to avoid errors & A \\
		Provide mechanisms to manage user's profiles & A \\
		Provide mechanisms for search by indexing, key or natural language & A \\
		The system's communication mechanisms match the needs of the users & A* \\
		Maintain UCD [user-centered design] attributes for interface graphical aspects & A \\
		Highlight cross-references between different types of data (e.g. customer issues, sales support, and marketing campaigns) & A* \\
		Register the date of last modification of documents to facilitate updating & A \\
		Provide easy-to-use authoring tools & A \\
		Supply different media channels for communication & A \\
		Introduce mechanisms to prevent usage errors & A \\
		Provide both synchronous and asynchronous communication tools & A \\
		The system conforms to platform conventions & A* \\
		Provide mechanisms for teaching-through-errors & A \\
		Clearly visualize employee performace & A* \\
		Even better than good error messages is a careful design which prevents a problem from occurring in the first place. & N \\
		Error messages should be expressed in plain language (no codes), precisely indicate the problem, and constructively suggest a solution. & N \\
		Even though it is better if the system can be used without documentation, it may be necessary to provide help and documentation. Any such information should be easy to search, focused on the user's task, list concrete steps to be carried out, and not be too large. & N \\
		Dialogues should not contain information which is irrelevant or rarely needed. Every extra unit of information in a dialogue competes with the relevant units of information and diminishes their relative visibility. & N \\
		Accelerators -- unseen by the novice user -- may often speed up the interaction for the expert user to such an extent that the system can cater to both inexperienced and experienced users. Allow users to tailor frequent actions. & N \\
		Make object, actions, and options visible. The user should not have to remember information from one part of the dialogue to another. Instructions for use of the system should be visible or easily retrievable whenever appropriate. & N \\
		Users should not have to wonder whether different words, situations, or actions mean the same thing. Follow platform conventions. & N \\
		The system should always keep users informed about what is going on, through appropriate feedback within reasonable time. & N \\
		The system should speak the users' language with words, phrases, and concepts familiar to the user, rather than system-oriented terms. Follow real-world conventions, making information appear in a natural and logical order. & N \\
		Users often choose system functions by mistake and will need a clearly marked "emergency exit" to leave the unwanted state without having to go through an extended dialogue. Support and undo and redo. & N \\
		The system provides appropriate filters to organize data & New \\
		The system has a dashboard which provides a quick glance of the current status & New \\
		The system displays appropriate information depending on the task at hand & New \\
		Help and documentation are immersed in the system, non-obtrusive, and ubiquitous & New \\
		The system allows for tailoring of the interface to an individual's workflow & New \\
		The terminology used by the system is consistent with the terminology of the user & S \\
		The various functions of the system can be identified by exploration & S \\
		The system is capable of supporting the different interaction styles of the various users & S \\
		Functionality can be found quickly and easily & S \\
		The information provided by the system is in real-time & S \\
		The ability of the system to be re-configured over a period of time & S \\
		The output style fits the type of data being displayed & S* \\
		The responses from the system are quick and efficient & S \\
		The system is easy to use & S \\
		The ease in which the system can be configured to a particular industry type & S \\
		The system supports efficient completion of tasks & S \\
		The system can guide the user through the correct sequence of transaction to complete a business process & S \\
		The system reduces intimidation and complexity by providing positive feedback and reinforcement, a clear path to execution, and a terminology that matches the users' language & S* \\
		The output provided provides clear visibility into the various other departments & S \\
		The capability of the system to support user-level customization & S \\
		The system automates routine and redundant tasks & S \\
		The alignment of the system to update existing business processes, and (or) to include new ones & S \\
		The results returned by a search are relevant to the information required by the user & S* \\
		The ability of the UI to be configured without affecting the underlying business logic of the system & S \\
		Information can be easily accessed & S \\
		The system supports improved information flow between the various organizational departments & S \\
		There is clarity in terms of the next sequence of transactions of steps & S \\
		It is easy to become skillful at using the system within a short amount of time & S \\
		The UI supports efficient and accurate navigation of the system & S \\
		The system improves user productivity & S \\
		Functionality to search for information that is available & S \\
		The information provided by the system is timely, accurate, complete and understandable & S \\
		The information presented supports informed decision making & S \\
		There is sufficient on-line help to support the learning process & S \\
		The visual layout is well designed & S \\
		The UI is intuitive & S \\
		A user can learn how to use the system without a long introduction & S \\
	\end{longtabu}
\end{singlespace}

\section{List Validated by CRM Users}
\label{appsec:users_list}
\begin{singlespace}
		\tabulinesep=_2mm^2mm
	\begin{longtabu} to \textwidth {X[1, p] c}
			\caption[List of heuristics developed during second phase]{List of heuristics developed during second phase; * heuristic is a reworded version of original; N = \citet{Nielsen1994a}, S = \citet{Singh2009}, A = \citet{Ardito2006}, New = newly developed heuristic}\\
			\toprule
			\textbf{Heuristic} & \textbf{Source} \\
			\midrule
		\endfirsthead
			
			\textbf{Heuristic} & \textbf{Source} \\
			\midrule
		\endhead
		
			\bottomrule
		\endlastfoot
		
		Clearly and constantly indicate system state & A \\
		Clearly visualize employee performance & A* \\
		Clearly visualize options and commands available & A \\
		Clearly visualize progress tracking & A \\
		Clearly visualize user workflow & A* \\
		Highlight cross-references between different types of data (e.g. customer issues, sales support, and marketing campaigns) & A* \\
		Insert mechanisms to make annotations & A \\
		Introduce mechanism to highlight errors and cues to avoid errors & A \\
		Introduce mechanisms to prevent usage errors & A \\
		Maintain UCD [user-centered design] attributes for interface graphical aspects & A \\
		Maximize adaptation of technology to the context of use & A \\
		Provide both synchronous and asynchronous communication tools & A \\
		Provide easy-to-use authoring tools & A \\
		Provide mechanisms for search by indexing, key or natural language & A \\
		Provide mechanisms for teaching-through-errors & A \\
		Provide mechanisms to manage user's profiles & A \\
		Register the date of last modification of documents to facilitate updating & A \\
		Supply different media channels for communication & A \\
		The system conforms to platform conventions & A* \\
		The system is customizable at the user level & A* \\
		The system's communication mechanisms match the needs of the users & A* \\
		Accelerators -- unseen by the novice user -- may often speed up the interaction for the expert user to such an extent that the system can cater to both inexperienced and experienced users. Allow users to tailor frequent actions. & N \\
		Dialogues should not contain information which is irrelevant or rarely needed. Every extra unit of information in a dialogue competes with the relevant units of information and diminishes their relative visibility. & N \\
		Error messages should be expressed in plain language (no codes), precisely indicate the problem, and constructively suggest a solution. & N \\
		Even better than good error messages is a careful design which prevents a problem from occurring in the first place. & N \\
		Even though it is better if the system can be used without documentation, it may be necessary to provide help and documentation. Any such information should be easy to search, focused on the user's task, list concrete steps to be carried out, and not be too large. & N \\
		Make object, actions, and options visible. The user should not have to remember information from one part of the dialogue to another. Instructions for use of the system should be visible or easily retrievable whenever appropriate. & N \\
		The system should always keep users informed about what is going on, through appropriate feedback within reasonable time. & N \\
		The system should speak the users' language with words, phrases, and concepts familiar to the user, rather than system-oriented terms. Follow real-world conventions, making information appear in a natural and logical order. & N \\
		Users should not have to wonder whether different words, situations, or actions mean the same thing. Follow platform conventions. & N \\
		Help and documentation are immersed in the system, non-obtrusive, and ubiquitous & New \\
		The system allows for tailoring of the interface to an individual's workflow & New \\
		The system displays appropriate information depending on the task at hand & New \\
		The system has a dashboard which provides a quick glance of the current status & New \\
		The system provides appropriate filters to organize data & New \\
		A user can learn how to use the system without a long introduction & S \\
		Functionality can be found quickly and easily & S \\
		Functionality to search for information that is available & S \\
		Information can be easily accessed & S \\
		It is easy to become skillful at using the system within a short amount of time & S \\
		The ability of the system to be re-configured over a period of time & S \\
		The ability of the UI to be configured without affecting the underlying business logic of the system & S \\
		The alignment of the system to update existing business processes, and (or) to include new ones & S \\
		The capability of the system to support user-level customization & S \\
		The ease in which the system can be configured to a particular industry type & S \\
		The information presented supports informed decision making & S \\
		The information provided by the system is in real-time & S \\
		The information provided by the system is timely, accurate, complete and understandable & S \\
		The output provided provides clear visibility into the various other departments & S \\
		The output style fits the type of data being displayed & S* \\
		The responses from the system are quick and efficient & S \\
		The results returned by a search are relevant to the information required by the user & S* \\
		The system automates routine and redundant tasks & S \\
		The system can guide the user through the correct sequence of transaction to complete a business process & S \\
		The system improves user productivity & S \\
		The system is capable of supporting the different interaction styles of the various users & S \\
		The system is easy to use & S \\
		The system reduces intimidation and complexity by providing positive feedback and reinforcement, a clear path to execution, and a terminology that matches the users' language & S* \\
		The system supports efficient completion of tasks & S \\
		The system supports improved information flow between the various organizational departments & S \\
		The terminology used by the system is consistent with the terminology of the user & S \\
		The UI [user interface] is intuitive & S \\
		The UI supports efficient and accurate navigation of the system & S \\
		The various functions of the system can be identified by exploration & S \\
		The visual layout is well designed & S \\
		There is clarity in terms of the next sequence of transactions of steps & S \\
		There is sufficient on-line help to support the learning process & S \\
	\end{longtabu}
\end{singlespace}

\section{Final, Unified List of Heuristics}
\label{appsec:final}
\begin{singlespace}
		\tabulinesep=_1mm^1mm
	\begin{longtabu} to \textwidth {l X[1, p] c}
			\caption{Final, unified list of heuristics for web-based CRM systems}\\
			\toprule
			\textbf{Category} & \textbf{Heuristic} & \textbf{Source} \\
			\midrule
		\endfirsthead
			
			\textbf{Category} & \textbf{Heuristic} & \textbf{Source} \\
			\midrule
		\endhead
		
			\bottomrule
		\endlastfoot
		
		Application Proactivity & Introduce mechanisms to prevent usage errors & A \\
		Application Proactivity & Maximize adaptation of technology to the context of use & A \\
		Application Proactivity & Provide mechanisms for teaching-through-er\-rors & A \\
		Application Proactivity & Provide mechanisms to manage user's profiles & A \\
		Application Proactivity & Register the date of last modification of documents to facilitate updating & A \\
		Application Proactivity & The system conforms to platform conventions & A \\
		Customization & The ability of the system to be re-configured over a period of time & S \\
		Customization & The ability of the UI to be configured without affecting the underlying business logic of the system & S \\
		Customization & The alignment of the system to update existing business processes, and (or) to include new ones & S \\
		Customization & The capability of the system to support user-level customization & S \\
		Customization & The ease in which the system can be configured to a particular industry type & S \\
		Customization & The system allows for tailoring of the interface to an individual's workflow & New \\
		Learnability & A user can learn how to use the system without a long introduction & S \\
		Learnability & Even though it is better if the system can be used without documentation, it may be necessary to provide help and documentation. Any such information should be easy to search, focused on the user's task, list concrete steps to be carried out, and not be too large. & N \\
		Learnability & Help and documentation are immersed in the system, non-obtrusive, and ubiquitous & New \\
		Learnability & It is easy to become skillful at using the system within a short amount of time & S \\
		Learnability & The system reduces intimidation and complexity by providing positive feedback and reinforcement, a clear path to execution, and a terminology that matches the users' language & S \\
		Learnability & The various functions of the system can be identified by exploration & S \\
		Learnability & There is sufficient on-line help to support the learning process & S \\
		Navigation & Accelerators -- unseen by the novice user -- may often speed up the interaction for the expert user to such an extent that the system can cater to both inexperienced and experienced users. Allow users to tailor frequent actions. & N \\
		Navigation & Even better than good error messages is a careful design which prevents a problem from occurring in the first place. & N \\
		Navigation & Functionality can be found quickly and easily & S \\
		Navigation & Functionality to search for information that is available & S \\
		Navigation & Information can be easily accessed & S \\
		Navigation & Make object, actions, and options visible. The user should not have to remember information from one part of the dialogue to another. Instructions for use of the system should be visible or easily retrievable whenever appropriate. & N \\
		Navigation & The results returned by a search are relevant to the information required by the user & S \\
		Navigation & The system can guide the user through the correct sequence of transaction to complete a business process & S \\
		Navigation & The system is capable of supporting the different interaction styles of the various users & S \\
		Navigation & The UI supports efficient and accurate navigation of the system & S \\
		Navigation & There is clarity in terms of the next sequence of transactions of steps & S \\
		Presentation & Clearly and constantly indicate system state & A \\
		Presentation & Clearly visualize employee performance & A \\
		Presentation & Clearly visualize options and commands available & A \\
		Presentation & Clearly visualize progress tracking & A \\
		Presentation & Clearly visualize user workflow & A \\
		Presentation & Dialogues should not contain information wh\-ich is irrelevant or rarely needed. Every extra unit of information in a dialogue competes with the relevant units of information and diminishes their relative visibility. & N \\
		Presentation & Error messages should be expressed in plain language (no codes), precisely indicate the problem, and constructively suggest a solution. & N \\
		Presentation & Introduce mechanism to highlight errors and cues to avoid errors & A \\
		Presentation & Maintain UCD [user-centered design] at\-trib\-utes for interface graphical aspects & A \\
		Presentation & The information presented supports informed decision making & S \\
		Presentation & The information provided by the system is timely, accurate, complete and understandable & S \\
		Presentation & The output provided provides clear visibility into the various other departments & S \\
		Presentation & The output style fits the type of data being displayed & S \\
		Presentation & The system is accessible and usable from mobile devices & New \\
		Presentation & The system is customizable at the user level & A \\
		Presentation & The system should always keep users informed about what is going on, through appropriate feedback within reasonable time. & N \\
		Presentation & The UI [user interface] is intuitive & S \\
		Presentation & The visual layout is well designed & S \\
		Task Support & Highlight cross-references between different ty\-pes of data (e.g. customer issues, sales support, and marketing campaigns) & A \\
		Task Support & Supply different media channels for communication & A \\
		Task Support & The information provided by the system is in real-time & S \\
		Task Support & The responses from the system are quick and efficient & S \\
		Task Support & The system allows for synchronization of its information with outside communication tools & New \\
		Task Support & The system automates routine and redundant tasks & S \\
		Task Support & The system displays appropriate information depending on the task at hand & New \\
		Task Support & The system has a dashboard which provides a quick glance of the current status & New \\
		Task Support & The system improves user productivity & S \\
		Task Support & The system is easy to use & S \\
		Task Support & The system provides appropriate filters to organize data & New \\
		Task Support & The system should speak the users' language with words, phrases, and concepts familiar to the user, rather than system-oriented terms. Follow real-world conventions, making information appear in a natural and logical order. & N \\
		Task Support & The system supports efficient completion of tasks & S \\
		Task Support & The system supports improved information flow between the various organizational departments & S \\
		Task Support & The terminology used by the system is consistent with the terminology of the user & S \\
		Task Support & Users often choose system functions by mistake and will need a clearly marked "emergency exit" to leave the unwanted state without having to go through an extended dialogue. Support and undo and redo. & N \\
		Task Support & Users should not have to wonder whether different words, situations, or actions mean the same thing. Follow platform conventions. & N \\
		User Activity & Insert mechanisms to make annotations & A \\
		User Activity & Provide both synchronous and asynchronous communication tools & A \\
		User Activity & Provide easy-to-use authoring tools & A \\
		User Activity & Provide mechanisms for search by indexing, key or natural language & A \\
		User Activity & The system's communication mechanisms mat\-ch the needs of the users & A \\
	\end{longtabu}
\end{singlespace}